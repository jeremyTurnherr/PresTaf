\section{Résumé du projet}

Notre projet s'inscrit dans la mis-à-jour de PresTaf, logiciel d'analyse de formules logique et de leur transformation en automate minimal. Nous intervenons ici dans la création d'un interfaçage Lua, ainsi qu'une reprise du code source PresTaf pour donner une version plus optimisé, plus claire et propre. Qui plus est, le Lua étant un langage basé sur le C++ cela offre la possibilité à l'utilisateur de surcharger des opérateurs. Ainsi il pourra utiliser les opérateurs pour écrire simplement ses formules logiques de façon simple.\\\par

Dans un second temps nous chercherons à intégrer de nouvelles logiques telle que la logique monadique ou encore l'interprétation des formules de Presburger en base -2.
