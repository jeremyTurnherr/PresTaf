%% Based on a TeXnicCenter-Template by Gyorgy SZEIDL.
%%%%%%%%%%%%%%%%%%%%%%%%%%%%%%%%%%%%%%%%%%%%%%%%%%%%%%%%%%%%%

%------------------------------------------------------------
%
\documentclass{article}%
%Options -- Point size:  10pt (default), 11pt, 12pt
%        -- Paper size:  letterpaper (default), a4paper, a5paper, b5paper
%                        legalpaper, executivepaper
%        -- Orientation  (portrait is the default)
%                        landscape
%        -- Print size:  oneside (default), twoside
%        -- Quality      final(default), draft
%        -- Title page   notitlepage, titlepage(default)
%        -- Columns      onecolumn(default), twocolumn
%        -- Equation numbering (equation numbers on the right is the default)
%                        leqno
%        -- Displayed equations (centered is the default)
%                        fleqn (equations start at the same distance from the right side)
%        -- Open bibliography style (closed is the default)
%                        openbib
% For instance the command
%           \documentclass[a4paper,12pt,leqno]{article}
% ensures that the paper size is a4, the fonts are typeset at the size 12p
% and the equation numbers are on the left side
%
\usepackage{amsmath}%
\usepackage{amsfonts}%
\usepackage{amssymb}%
\usepackage{graphicx}
\usepackage[francais]{babel}
\usepackage{hyperref}
%-------------------------------------------
\newtheorem{theorem}{Theorem}
\newtheorem{acknowledgement}[theorem]{Acknowledgement}
\newtheorem{algorithm}[theorem]{Algorithm}
\newtheorem{axiom}[theorem]{Axiom}
\newtheorem{case}[theorem]{Case}
\newtheorem{claim}[theorem]{Claim}
\newtheorem{conclusion}[theorem]{Conclusion}\newtheorem{condition}[theorem]{Condition}
\newtheorem{conjecture}[theorem]{Conjecture}
\newtheorem{corollary}[theorem]{Corollary}
\newtheorem{criterion}[theorem]{Criterion}
\newtheorem{definition}[theorem]{Definition}
\newtheorem{example}[theorem]{Example}
\newtheorem{exercise}[theorem]{Exercise}
\newtheorem{lemma}[theorem]{Lemma}
\newtheorem{notation}[theorem]{Notation}
\newtheorem{problem}[theorem]{Problem}
\newtheorem{proposition}[theorem]{Proposition}
\newtheorem{remark}[theorem]{Remark}
\newtheorem{solution}[theorem]{Solution}
\newtheorem{summary}[theorem]{Summary}
\newenvironment{proof}[1][Proof]{\textbf{#1.} }{\ \rule{0.5em}{0.5em}}

\begin{document}

\title{M\'emoire interm\'ediaire PresTaf}
\author{Bourgeois Adrien, Marbois Bryce, Roque Maxime, Turhneer J\'er\'emy\thanks{Si l'on veut faire des annotations.}
\\Universit\'e UFR Coll\'egium des Sciences et Technique d'Orl\'eans-la-Source}
\date{\today}
\maketitle

\begin{abstract}
R\'esum\'e
\end{abstract}

\tableofcontents

\section{R\'esum\'e du projet}

\section{Domaine}

Math\'ematiques, Logique, Th\'eorie des langages, Th\'eorie des graphes.

\section{Analyse de PresTaf}

PresTaf est programme cod\'e par Couvreur, qui prend des formules de Presburger en entr\'ees et les r\'esous. Les automates g\'en\'er\'es sont d\'eterministes et finis mais non minimaux. Il faut donc les minimiser, et pour se faire il faut utiliser l'algorithme d'Hopcroft. La o\`u M.Couvreur avait fait une impl\'ementation intelligence de l'algorithme, mais elle malgr\'e son temps d'execution tr\`es performant, il n'est pas facile a lire, et l'impl\'ementation est assez vieille.

\section{Besoins non fonctionnels}

\begin{itemize}
\item Compr\'ehension des formules de Presburger
\item Compr\'ehension des formules monadiques / lash
\item Th\'eorie des langages, compr\'ehension des automates
\end{itemize}

\section{Besoins fonctionnels}

\begin{itemize}
\item Interfa\c{c}age en Lua pour permettre une utilisation simple de PresTaf.
\item Algorithme de minimisation d'automate.
\item Revision g\'en\'eral de PresTaf
\end{itemize}

\section{Prototypes et tests pr\'eparatoires}

\begin{itemize}
  \item JavaLua
  \item Impl\'ementation Python de l'algorithme de Blum (variante de l'algorithme d'Hopcroft)
\end{itemize}


\section{Planning}

\begin{enumerate}
\item Impl\'emation de l'algorithme de Blum.
\item R\'evision du code PresTaf.
\item Iterfa\c{c}age en Lua.
\item R\'edaction du m\'emoire interm\'ediaire
\end{enumerate}

\begin{thebibliography}{9}                                                                                                %
beg\bibitem {KarelRektorys}Rektorys, K., \textit{Variational methods in Mathematics,
Science and Engineering}, D. Reidel Publishing Company,
Dordrecht-Hollanf/Boston-U.S.A., 2th edition, 1975

\bibitem {Bertoti97} \textsc{Bert\'{o}ti, E.}:\ \textit{On mixed variational formulation
of linear elasticity using nonsymmetric stresses and displacements}, International
Journal for Numerical Methods in Engineering., \textbf{42}, (1997), 561-578.

\bibitem {Szeidl2001} \textsc{Szeidl, G.}:\ \textit{Boundary integral equations for
plane problems in terms of stress functions of order one}, Journal of Computational and
Applied Mechanics, \textbf{2}(2), (2001), 237-261.

\bibitem {Carlson67}  \textsc{Carlson D. E.}:\ \textit{On G\"{u}nther's stress functions
for couple stresses}, Quart. Appl. Math., \textbf{25}, (1967), 139-146.
\end{thebibliography}


\appendix

\section{Premi\`ere annexe}

\end{document}
