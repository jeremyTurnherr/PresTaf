%% Based on a TeXnicCenter-Template by Gyorgy SZEIDL.
%%%%%%%%%%%%%%%%%%%%%%%%%%%%%%%%%%%%%%%%%%%%%%%%%%%%%%%%%%%%%

%------------------------------------------------------------
%
\documentclass[12pt]{article}%
%Options -- Point size:  10pt (default), 11pt, 12pt
%        -- Paper size:  letterpaper (default), a4paper, a5paper, b5paper
%                        legalpaper, executivepaper
%        -- Orientation  (portrait is the default)
%                        landscape
%        -- Print size:  oneside (default), twoside
%        -- Quality      final(default), draft
%        -- Title page   notitlepage, titlepage(default)
%        -- Columns      onecolumn(default), twocolumn
%        -- Equation numbering (equation numbers on the right is the default)
%                        leqno
%        -- Displayed equations (centered is the default)
%                        fleqn (equations start at the same distance from the right side)
%        -- Open bibliography style (closed is the default)
%                        openbib
% For instance the command
%           \documentclass[a4paper,12pt,leqno]{article}
% ensures that the paper size is a4, the fonts are typeset at the size 12p
% and the equation numbers are on the left side
%
\usepackage{amsmath}%
\usepackage{amsfonts}%
\usepackage{amssymb}%
\usepackage{graphicx}
\usepackage{subfigure}
\usepackage[francais]{babel}
\usepackage[T1]{fontenc}
\usepackage[utf8]{inputenc}
\usepackage{tikz}
\usetikzlibrary{calc,arrows,automata,positioning}
\usepackage{tkz-graph}
\usepackage{cite}
\usepackage{listings}
\usepackage{fancyhdr}
\usepackage{xcolor}
\usepackage{hyperref}
\hypersetup{     
	colorlinks,     
	linkcolor={black},     
	citecolor={black},     
	urlcolor={black} 
}
\usepackage[toc,section=section,nonumberlist]{glossaries}
\usepackage{url}
\makeglossaries
%-------------------------------------------
\newtheorem{theorem}{Theorem}
\newtheorem{acknowledgement}[theorem]{Acknowledgement}
\newtheorem{algorithm}[theorem]{Algorithm}
\newtheorem{axiom}[theorem]{Axiom}
\newtheorem{case}[theorem]{Case}
\newtheorem{claim}[theorem]{Claim}
\newtheorem{conclusion}[theorem]{Conclusion}\newtheorem{condition}[theorem]{Condition}
\newtheorem{conjecture}[theorem]{Conjecture}
\newtheorem{corollary}[theorem]{Corollary}
\newtheorem{criterion}[theorem]{Criterion}
\newtheorem{definition}[theorem]{Definition}
\newtheorem{example}[theorem]{Example}
\newtheorem{exercise}[theorem]{Exercise}
\newtheorem{lemma}[theorem]{Lemma}
\newtheorem{notation}[theorem]{Notation}
\newtheorem{problem}[theorem]{Problem}
\newtheorem{proposition}[theorem]{Proposition}
\newtheorem{remark}[theorem]{Remark}
\newtheorem{solution}[theorem]{Solution}
\newtheorem{summary}[theorem]{Summary}
\newenvironment{proof}[1][Proof]{\textbf{#1.} }{\ \rule{0.5em}{0.5em}}

%--------------------------------------------------------------------%
%         POUR DEFINIR DES ENTRÉES DU GLOSSAIRE LE FAIRE ICI         %
%              POUR LES ACRONYMES C'EST AU MÊME ENDROIT              %
%--------------------------------------------------------------------%

\longnewglossaryentry{presburger}{
	name=Arithmétique de Presburger
}
{
	L'arithmétique de Presburger à été introduite par Moj\.{z}esz Presburger en 1929. Cette arithmétique du premier ordre dispose de deux constantes 0 et 1 ainsi qu'un symbole binaire +. Ce langage est limité aux entiers naturels et est défini par les lois suivantes :
	\begin{enumerate}
		\item $\forall x, \neg(0 = x + 1)$
		\item $\forall x, \forall y, x + 1 = y + 1 \rightarrow x = y $
		\item $\forall x, x + 0 = x$
		\item $\forall x, \forall y, x + (y + 1) = (x + y) + 1$
		\item $\forall P(x, y_1, \ldots, y_n) \in$ Formule du premier ordre,\\
		$\forall y_1 \ldots \forall y_n [(P(0, y_1, \ldots,y_n) \vee \forall x(P(x, y_1, \ldots, y_n) \rightarrow \\P(x + 1, y_1, \ldots, y_n))) \rightarrow \forall y P(y, y_1, \ldots, y_n)]$
	\end{enumerate}
}

\longnewglossaryentry{monadique}{
	name=Logique monadique du second ordre
} {
	aussi connu sous le nom de \emph{Monadic Second Order} ou \emph{MSO}, est notamment utilisé dans un autre programme de M.Couvreur : VeriTaf. VeriTaf permet de vérifier des formules CTL (Computation Tree Logic) et des formules LTL (Linear Temporal Logic)
}

%--------------------------------------------------------------------%
%         POUR DEFINIR DES ENTRÉES DU GLOSSAIRE LE FAIRE ICI         %
%              POUR LES ACRONYMES C'EST AU MÊME ENDROIT              %
%--------------------------------------------------------------------%

\renewcommand{\headrulewidth}{1pt}
\fancyhead[C]{} 
\fancyhead[L]{\leftmark}
\fancyhead[R]{\large{\textbf{PresTaf}}}

\renewcommand{\footrulewidth}{1pt}
\fancyfoot[C]{} 
\fancyfoot[L]{\textbf{\today}}
\fancyfoot[R]{\textbf{page \thepage}}

\begin{document}
\pagestyle{empty}

\begin{titlepage}
\begin{center}
\noindent{\Huge \textbf{PresTaf}}\\
\vspace{0.5cm}
%\noindent{\LARGE \textbf{Une bibliothèque d'automate}}\\
\vspace{2cm}
{\Large Bourgeois Adrien, Marbois Bryce, Roque Maxime, Turnherr Jérémy
\\Université UFR Collégium des Sciences et Technique
\\Orléans-la-Source}\\
\vfill
{\large\today}
\end{center}
\end{titlepage}

\clearpage

\tableofcontents

\cleardoublepage

\pagestyle{fancy}

\section{Résumé du projet}

Notre projet s'inscrit dans la mis-à-jour de PresTaf, logiciel d'analyse de formules logique et de leur transformation en automate minimal. Nous intervenons ici dans la création d'un interfaçage Lua, ainsi qu'une reprise du code source PresTaf pour donner une version plus optimisé, plus claire et propre. Qui plus est, le Lua étant un langage basé sur le C++ cela offre la possibilité à l'utilisateur de surcharger des opérateurs. Ainsi il pourra utiliser les opérateurs pour écrire simplement ses formules logiques de façon simple.\\\par

Dans un second temps nous chercherons à intégrer de nouvelles logiques telle que la logique monadique ou encore l'interprétation des formules de Presburger en base -2.


\section{Domaine}

Le logiciel sur lequel nous nous appuyions pour notre travail initial est PresTaf. Il implemente en Java la logique \gls{presburger}\cite{ginsburg1966semigroups}. Il existe des logiciels concurrent travaillant avec d'autres logiques telle que la \gls{monadique}\cite{KlaEtAl:Mona} avec des logiciel comme Mona\cite{monamanual2001}, ou la logique arithmétique de Presburger et les d'autre logiques sur les mots infini avec Lash\cite{lash}.


\section{Analyse de l'existant}

PresTaf est un programme codé par M. Jean-Michel Couvreur, qui prend des formules de Presburger en entrées et les résout à l'aide d'automates minimaux. Tout d'abord il génère des automates déterministes et finis mais non minimaux. Il faut donc les minimiser, et pour se faire PresTaf utilise un algorithme d'Hopcroft modifié. L'ensemble des transitions menant de l'état initial vers un état final est solution de la formule. En outre si l'état final est l'état \emph{zero} alors il n'y a aucune solution et si l'état final est l'état \emph{one} alors la formule est une tautologie.\\\par

\clearpage Mona, est une bibliothèque C qui résout des formules monadiques. La où PresTaf n'implémente à ce jour que la logique arithmétique de Presburger, la logique monadique pourrait être implémentée dans le futur.\\\par

Lash\cite{lash} est une bibliothèque C qui résout des formules de Presburger mais à la différence de PresTaf, il fonctionne sur des automates infinis. Cette différence induit une importante baisse de performances. En effet pour les mêmes formules, PresTaf était bien plus rapide à s'executer que Lash\cite{DBLP:conf/wia/Couvreur04}.\\\par

Prenons une formule simple, comme $x = y$. Dans cette formule, deux variables seront lues simultanément en partant du bit de poids faible, mais imaginons la bibliothèque PresTaf comme une machine de Turing à une seule bande uni-infini. D'instinct, on souhaiterait lire deux par deux les bits mais ce n'est pas ce que fait la bibliothèque, à la place c'est comme si la bande alternait les bits de $x$ avec ceux de $y$.\\\par
Prenons par exemple $x = 100110$ et $y = 101$, pour commencer PresTaf ajoute autant de 0 devant que nécessaire pour que $x$ et $y$ fassent la même taille, donc on a $y'=000101$ pour plus de lisibilité nous allons noter les bits correspondant à $x$ comme ceci : $b_x$ et respectivement $b_y$. Ainsi la bande construite sera la suivante :

\begin{figure}[h]
\begin{tikzpicture}
\draw[thick] (0, 0) -- (13.5, 0);
\draw[thick] (0, 1) -- (13.5, 1);
\draw[thick] (0, 0) -- (0, 1);

\foreach \x in {1,...,12} {
	\draw[thick] (\x, 0) -- (\x, 1);
}

\node at (0.5,0.5) {\large $0_x$};
\node at (1.5,0.5) {\large $1_y$};

\node at (2.5,0.5) {\large $1_x$};
\node at (3.5,0.5) {\large $0_y$};

\node at (4.5,0.5) {\large $1_x$};
\node at (5.5,0.5) {\large $1_y$};

\node at (6.5,0.5) {\large $0_x$};
\node at (7.5,0.5) {\large $0_y$};

\node at (8.5,0.5) {\large $0_x$};
\node at (9.5,0.5) {\large $0_y$};

\node at (10.5,0.5) {\large $1_x$};
\node at (11.5,0.5) {\large $0_y$};

\draw[-latex] (0.5, 1.5) -- (0.5, 1);

\end{tikzpicture}

\caption{Bande d'entrée, avec la tête de lecture en $0_x$}
\end{figure}

Comparons maintenant les automates produits par cette formule tels qu'ils seraient dessinés en lisant les bits deux à deux selon cette bande uni-infini. \clearpage

\begin{figure}[h]

\begin{tikzpicture}

\node[circle,draw, accepting] (A) at (0, 0)  {$q_0$};
\node[circle,draw, accepting] (B) at (0, -2)  {$zero$};

\path[->, thick] (0, 1) edge (A)
(A) edge [bend right] node[left]{$0_x1_y$} (B)
(A) edge [bend left] node[right]{$1_x0_y$} (B)
(A) edge [loop left] node[left]{$0_x0_y$} (A)
(A) edge [loop right] node[right]{$1_x1_y$} (A);

\node[circle,draw, accepting] (C) at (8.5, 0)  {$q_0$};
\node[circle,draw] (D) at (7, -2)  {$q_1$};
\node[circle,draw,accepting] (E) at (10, -2)  {$q_2$};
\node[circle,draw, accepting] (F) at (8.5, -4)  {$zero$};

\path[->, thick] (8.5, 1) edge (C)
(C) edge [bend left] node[right]{$0_x$} (E)
(E) edge [bend left] node[right]{$0_y$} (C)
(C) edge [bend left] node[right]{$1_x$} (D)
(D) edge [bend left] node[left]{$1_y$} (C)
(D) edge node[left]{$0_y$} (F)
(E) edge node[right]{$1_y$} (F);

\end{tikzpicture}


\caption{À gauche l'automate "instinctif", à droite l'automate produit par PresTaf}
\end{figure}

Le nombre d'état s'en voit multiplié comme on ne lit pas simultanément les bits. Revenons-en à notre bande $x = 100110$ et $y = 101$, ici, les nombres ne sont pas égaux. Si sur l'automate de gauche le résultat est immédiat, pour l'automate de droite il va d'abord se déplacer dessus avant d'arriver en 0. Cette écriture en "alternance" peut rendre rapidement les automates très grands, et peu lisibles, cependant ceci permet un alphabet unique de taille 2.\\\par

En effet avec l'exemple à deux variables, si nous lisions les bits simultanément on aurait l'alphabet suivant : $\Sigma = \{00, 01, 10, 11\}$, pour une formule à trois variables nous aurions $|\Sigma| = 8$, et ainsi de suite. Ceci nous donnerait un alphabet de taille exponentielle en la taille de l'entrée. Si nous avions $n$ variables, nous aurions $|\Sigma| = 2^n$. En prenant un tel alphabet, la minimisation d'automates serait beaucoup plus complexe, plus longue et plus couteuse en espace. Qui plus est, en ayant un alphabet unique ceci facilite grandement les prédictions en espace et en mémoire. Tandis qu'un alphabet à tailles variables ajoute un paramètre de compléxité qui n'est pas désiré.


\section{Besoins fonctionnels}

\subsection{Prototype papier}

Le prototype qui suit serait un fichier Lua qui se servirait de la logique $Presburger$ (cette logique étant elle-même codée en Lua).

\begin{lstlisting}[mathescape=true, frame=single]
pres = require('Presburger') // Choix de la logique

local x = variable('x') // Declaration d'une variable
local y = variable('y')
local f = equals(y + integer(1), x)

// Pour exporter l'automate
f:todot("f.dot")
\end{lstlisting}

Pour lancer ce fichier Lua il faudrait passer par un fichier jar que l'on appellera $prestaf.jar$. Pour lancer le jar et le fichier lua il faudrait faire la commande suivante :

\begin{lstlisting}[mathescape=true, frame=single]
java -jar prestaf.jar fichier.lua
\end{lstlisting}

Le fichier f.dot ressemblerait à :

\begin{tikzpicture}
\node[circle,draw]           (A) at (0.75, 6)  {$q_0$};
\node[circle,draw]           (B) at (0, 4.5)   {$q_1$};
\node[circle,draw,accepting] (C) at (1.5, 4.5) {$q_2$};
\node[circle,draw,accepting] (D) at (2, 3)     {$q_3$};
\node[circle,draw]           (E) at (2, 1.5)   {$q_4$};
\node[circle,draw,accepting] (F) at (0.75, 0)  {$zero$};

\path[->] (0.75, 7) edge (A)
(A) edge [bend right] node[above]{0} (B)
(B) edge [bend right] node[right]{1} (A)
(A) edge [bend left] node[right]{1} (C)
(B) edge [bend right] node[right]{0} (F)
(C) edge [bend right] node[right]{1} (F)
(C) edge [bend right] node[right]{0} (D)
(D) edge [bend right] node[right]{0} (C)
(D) edge [bend right] node[right]{1} (E)
(E) edge [bend right] node[right]{1} (D)
(E) edge [bend left] node[right]{0} (F);
\end{tikzpicture}

\subsection{Fonctions}

La priorité des fonctions varie de 1 à 5, du plus important au moins important, sachant que la priorité 5 correspond à une fonctionnalité optionnelle.

\subsubsection{Bilbiothèque d'automate générique}

\paragraph{Description :} La bibliothèque PresTaf est générique et doit accepter toutes sortes d'automates.

\paragraph{Justification :} L'utilisateur aura la possibilité d'implémenter ses propres logiques, la bibliothèque doit donc accepter toutes logiques. En effet PresTaf sera une bibliothèque d'automates, permettant de minimiser un automate, de faire des intersections, des unions, etc. Il ne faut donc pas que PresTaf soit ciblé sur Presburger, mona ou une quelconque autre logique.

\paragraph{Priorité :} 1\\

\rule{\linewidth}{1pt}

\subsubsection{Minimisation d'automate}

\paragraph{Description :} Ensemble de fonctions qui prennent un automate (fini, complet) et déterministe en entrée et retourne l'automate minimal équivalent.

\paragraph{Justification :} Besoin initial.

\paragraph{Priorité :} 1\\

\rule{\linewidth}{1pt}

\subsubsection{Interfaçage Lua}

\paragraph{Description :} Permet le codage des automates en Lua, ainsi que l'utilisation de chaques fonctions qui seront ensuite exécutées en Java.

\paragraph{Justification :} Le lua est un langage de script simple à prendre en main et qui permet facilement d'écrire des automates et d'utiliser des fonctions.

\paragraph{Priorité :} 2\\

\rule{\linewidth}{1pt}

\subsubsection{Portabilité du code}

\paragraph{Description :} Windows, MacOS, Linux

\paragraph{Justification :} Comme java est un langage portable executé via la Java Virutal Machine (JVM), et que LuaJava est executé via java il embarque sa propre machine virtuelle, en théorie le code sera donc portable sur tous les systèmes d'exploitation. En dehors de la portabilité du code, Windows MacOs et Linux sont les principaux systèmes d'exploitation, il est donc important d'avoir un code portable pour chaque machine pour faciliter l'accès.

\paragraph{Priorité :} 1\\

\rule{\linewidth}{1pt}

\subsubsection{Optimisation}

\paragraph{Description :} La bilbiothèque d'automates PresTaf doit être rapide à s'exécuter.

\paragraph{Justification :} L'utilisateur n'aura pas le temps d'attendre quelques dizaines de minutes que son automate soit généré. Il voudra obtenir son résultat rapidement.

\paragraph{Priorité :} 5


\section{Besoins non fonctionnels}

\subsection{Fonctions}

\subsubsection{Transformation de formules arithmétiques de Presburger}

\paragraph{Description :} Ensemble de fonctions qui prennent en entrée une formule arithmétique et retourne l'automate acceptant cette formule.

\paragraph{Justification :} La bilbiothèque PresTaf n'implémentera pas d'elle-même une logique. Ainsi l'implémentation de la logique de Presburger en script Lua, permettra de fournir une démo à l'utilisateur. Il aurait un aperçu des bonnes pratiques à avoir, les méthodes qu'il se doit d'implémenter, et des fonctionnalités présentes.

\paragraph{Priorité :} 1\\

\rule{\linewidth}{1pt}

\subsubsection{Logique monadique du second ordre}

\paragraph{Description :} Il s'agit d'une logique du second ordre, c'est-à-dire qu'un prédicat peut avoir en argument un autre prédicat, mais celui-ci ne peut pas avoir un troisième prédicat en argument (arité un). De plus dans le cadre de la logique monadique du second ordre les quantificateurs ne peuvent être utilisés que pour les variables des prédicats du premier ordre (de type Presburger par exemple).

\paragraph{Justification :} La logique de Presburger est moins complète que la logique Monadique, puisque la logique Monadique propose une notion de successeur, donc en implémentant la logique Monadique dans une autre bibliothèque Lua, on fournirait d'avantage de démo à l'utilisateur.

\paragraph{Priorité :} 3\\

\rule{\linewidth}{1pt}

\subsubsection{Acceptation de formules en base -2}

\paragraph{Description :} La base -2 est défini par : $ \sum\limits_{i=0}^n (-2)^i * k_i$. Par exemple $2_{decimal} = 0 * (-2)^0 + 1 * (-2)^1 \Rightarrow 2_{decimal} = -2_{base - 2}$. Pour déterminer un nombre en base -2, il suffit de determiner son écriture binaire et ensuite d'appliquer le calcule base -2. Si l'on a le nombre $10110010_{binaire}$ alors on aura en base -2 : $0 * (-2)^0 + 1 * (-2)^1 + 0 * (-2)^2 + 0 * (-2)^3 + 1 * (-2)^4 + 1 * (-2)^5 + 0 * (-2)^6 + 1 * (-2)^7 = -146_{base - 2}$

\paragraph{Justification :} Il serait intéressant de permettre à l'utilisateur d'utiliser cette bilbiothèque avec diverses bases, surtout la base -2.

\paragraph{Priorité :} 5


\section{Prototypes et tests préparatoires}

\subsection{Blum}

Voici un exemple que nous avons testé avec l'automate initial et l'objectif.

\begin{tikzpicture}[->,>=stealth',shorten >=1pt,auto,node distance=1.5cm,
                    semithick]

\node[circle,draw]           (A) at (2.25, 7.5) {$q_0$};
\node[circle,draw]           (B) at (2.25, 6)   {$q_1$};
\node[circle,draw]           (C) at (1.5, 4.5)  {$q_2$};
\node[circle,draw]           (D) at (3, 4.5)    {$q_3$};
\node[circle,draw]           (E) at (1.5, 3)    {$q_4$};
\node[circle,draw]           (F) at (3, 3)      {$q_5$};
\node[circle,draw]           (G) at (0, 1.5)    {$q_6$};
\node[circle,draw,accepting] (H) at (1.5, 1.5)  {$q_7$};
\node[circle,draw,accepting] (I) at (3, 1.5)    {$q_8$};
\node[circle,draw]           (J) at (4.5, 1.5)  {$q_9$};
\node[circle,draw,accepting] (K) at (0, 0)      {$q_{10}$};
\node[circle,draw,accepting] (L) at (4.5, 0)    {$q_{11}$};

\node at (6, 3.75) {$\Longrightarrow$};

\node[circle,draw]           (M) at (7.5, 1.5)       {$q_4$};
\node[circle,draw,accepting] (N) at (9, 0)           {$q_5$};
\node[circle,draw]           (O) at (9, 3)           {$q_3$};
\node[circle,draw]           (P) [above of=O]        {$q_2$};
\node[circle,draw]           (Q) [above of=P]        {$q_1$};
\node[circle,draw]           (R) [above of=Q]        {$q_0$};

\node at (2.25, 9) {Automate initial};
\node at (9, 9) {Automate minimal};

\path[->] (A) edge [right] node{0} (B)
(B) edge [bend right] node[above]{1} (C)
(B) edge [bend left] node[above]{0} (D)
(C) edge [left] node{0} (E)
(D) edge [right] node{0} (F)
(E) edge [bend right] node[above]{1} (G)
(E) edge [right] node{0} (H)
(F) edge [bend left] node[above]{1} (J)
(F) edge [left] node{0} (I)
(G) edge [left] node{0} (K)
(J) edge [right] node{0} (L)
(2.25, 8.5) edge (A)
(9, 8.5) edge (R)
(R) edge[right] node{0} (Q)
(Q) edge[bend right] node[left]{1} (P)
(Q) edge[bend left] node[right]{0} (P)
(P) edge[left] node{0} (O)
(O) edge[left] node{0} (N)
(O) edge[bend right] node[left]{1} (M)
(M) edge[bend right] node[left]{0} (N);

\end{tikzpicture}


\section{Exemple de fonctionnement}

\subsection{Un logiciel souple}



\begin{lstlisting}[mathescape=true, frame=single]
pres = require('Presburger')

local x = variable('x')
local y = variable('y')

local f = x '=' 2 * y 

f:todot("f.dot")
\end{lstlisting}

Le ficher f.dot est le suivant :

\vspace{0.5cm}

\begin{tikzpicture}

\node[circle,draw,accepting] (A) at (0, 4.5)    {$q_0$};
\node[circle,draw,accepting] (B) at (-1.5, 3)   {$q_1$};
\node[circle,draw]           (C) at (-1.5, 1.5) {$q_2$};
\node[circle,draw,accepting] (D) at (0, 0)      {$zero$};

\path[->] (0, 5.5) edge (A)
(A) edge [bend right] node[above]{0} (B)
(B) edge [bend right] node[right]{0} (A)
(B) edge [bend right] node[left]{1}  (C)
(C) edge [bend right] node[right]{1} (B)
(C) edge [bend right] node[above]{0} (D)
(A) edge [bend left]  node[right]{1} (D);
\end{tikzpicture}

Le résultat obtenus avec l'ancienne version de PresTaf est bien entendu la même. En revanche le Lua contient une différence majeure. Si l'on écrit le script suivant :

\begin{lstlisting}[mathescape=true, frame=single]
pres = require('Presburger')

local x = variable('x')
local y = variable('y')

local f = y '=' 2 * x 

f:todot("f.dot")
\end{lstlisting}

Le fichier générer sera différent : \vspace{0.5cm}

\begin{tikzpicture}

\node[circle,draw,accepting] (A) at (0, 4.5)    {$q_0$};
\node[circle,draw,accepting] (B) at (-1.5, 3)   {$q_4$};
\node[circle,draw]           (C) at (1, 3)      {$q_3$};
\node[circle,draw]           (D) at (1.5, 1.5)  {$q_1$};
\node[circle,draw]           (E) at (1, 0)      {$q_5$};
\node[circle,draw]           (F) at (3, 0)      {$q_2$};
\node[circle,draw,accepting] (G) at (0, -1.5)   {$zero$};

\path[->] (0, 5.5) edge (A)
(A) edge [bend right] node[above]{0} (B)
(B) edge [bend right] node[right]{0} (A)
(A) edge              node[left] {1} (C)
(F) edge [bend right] node[right]{1} (A)
(C) edge [bend right] node[left]{1} (G)
(C) edge              node[right]{0} (D)
(D) edge [bend left] node[right]{1} (E)
(E) edge [bend left] node[left]{1} (D)
(B) edge             node[left]{1} (G)
(D) edge             node[above]{0} (F)
(F) edge             node[below]{0} (G)
(E) edge             node[below]{0} (G);
\end{tikzpicture}

Ce qui permet de traiter les variables dans l'ordre désiré, ce qui laisse une plus grande marge de man\oe{}uvre.\\\par

Pour utiliser la bibliothèque Presburger l'utilisateur doit utiliser la fonction $require()$ de lua pour appeler les fichiers Term et Presburger.\\\par

Il existe les fonctions $variable(chaine)$ et $integer(nombre)$ qui permettent de créer respectivement une variable et un nombre. Ces deux élements peuvent ensuite être manipulés à l'aide de plusieurs fonctions qui implémentent la logique de Presburger.\\\par

La plupart de ces fonctions peuvent être également appelées directement par des opérateurs surchargés, de manière à ce que l'utilisateur puisse avoir une écriture plus lisible et concise pour écrire de très longue formules Presburger. On trouve ainsi avec les opérateurs:
\begin{itemize}
 \item les opérations $factor()$ *, $plus()$ + et $minus()$ -,
 \item les quantificateurs $\_A()$ 'A' et $\_E()$ 'E',
 \item les comparateurs $equals()$ $'='$, $not\_Equals()$ $!=$, $greater()$ $>$, $greaterEquals()$ $>=$, $lessEquals()$ $<=$,$less()$ $<$,
 \item les opérateurs logiques $and()$ $\&\&$, $or()$ $||$, $not()$, $equiv()$ $<->$  et $imply()$ $->$ .
\end{itemize}
Voici un exemple d'utilisation pour la formule $6x + 4 = 1$\\
 
\begin{lstlisting}[mathescape=true, frame=single]
pres = require('Presburger')
local x = variable('x')
local f=(6*x+integer(4))'='(integer(1))
f:todot("f.dot")
\end{lstlisting}

Pour exprimer qu'une formule est toujours vraie ou fausse, PresTaf renvoit respectivement les états one et zero.
\begin{figure}[h]
\begin{lstlisting}[mathescape=true, frame=single]
pres = require('Presburger')
local x = variable('x')
local f=(_E(x,(x)'='(integer(1)))
f:todot("f.dot")
\end{lstlisting}

\begin{tikzpicture}[->,>=stealth',shorten >=1pt,auto,node distance=1.5cm,
                    semithick]

\node[circle,draw]           (A) at (0, 0) {$One$};

\end{tikzpicture}
\caption{Formule toujours vraie}
\end{figure}


\begin{lstlisting}[mathescape=true, frame=single]
pres = require('Presburger')
local x = variable('x')
local f=(_A(x,(x)'='(integer(1)))
f:todot("f.dot")
\end{lstlisting}

\begin{figure}[h]
\begin{tikzpicture}[->,>=stealth',shorten >=1pt,auto,node distance=1.5cm,
                    semithick]

\node[circle,draw]           (A) at (2, 2) {$Zero$};

\end{tikzpicture}
\caption{Formule toujours fausse}
\end{figure}




\section{Architecture}

L'architecture de notre logiciel est composée de 4 couches:
\begin{enumerate}
	\item La bibliothèque PresTaf qui permet de manipuler des automates et d'éffectuer des opérations dessus. Nous avons créé une interface en java qui permet d'accéder plus simplement aux fonctions de la bibliothèque.
	\item La première couche Lua est composée des différentes bibliothèques de logique utilisant la bibliothèque PresTaf. Nous avons développé la bibliothèque Presburger.
	\item La deuxième couche Lua contient les différents fichiers des utilisateurs des bibliothèques de logique.
	\item C'est le launcher, developpé en Java, qui permet de démarrer l'application avec une unique commande tout en permettant d'utiliser du java et du Lua.
\end{enumerate}

Ainsi, la première et quatrième couches de l'application font parties de la bibliothèque PresTaf. Il y a deux sortes d'utilisateurs. En effet, une partie des utilisateurs utiliseront directement notre bibliothèque pour développer leur propre bibliothèque de logique, tandis que l'autre partie des utilisateurs utiliseront les bibliothèques de logique. Un utilisateur peut également être utilisateur de sa propre bibliothèque. Le premier groupe d'utilisateur travaillera essentiellement sur la 2\textsuperscript{ème} couche de l'application, tandis que le second groupe travaillera sur la 3\textsuperscript{ème} couche.\\\par

Lua et Java n'étant normalement pas compatible, pour pouvoir interfacer Lua et Java il nous a fallu utiliser une librairie : LuaJava. Cette bibliothèque permet au Java d'executer du Lua et d'altérer sa pile, en ajoutant et modifiant des variables.\\

\begin{figure}[h]
\begin{tikzpicture}

\draw (5, 0.25) rectangle (7, 1.25);
\draw[color=blue!25, ultra thick] (5.25, 1) -- (6, 1);
\draw (6.5, 1) node{Java};
\draw[color=red!25, ultra thick] (5.25, 0.5) -- (6, 0.5);
\draw (6.5, 0.5) node{Lua};

\draw[fill=blue!25] (0,0) rectangle (7,-1);
\draw (3.5, -0.5) node{PrestafLauncher};

\draw[->,thick] (3.5, -1) -- (3.5, -2);

\draw[fill=red!25] (0, -2) rectangle (7, -3);
\draw (3.5, -2.5) node{LuaInterface};

\draw[fill=red!25] (0, -4) rectangle (2.5, -5);
\draw[fill=red!25] (4.5, -4) rectangle (7, -5);
\draw (1.25, -4.5) node{Logique};
\draw (5.75, -4.5) node{Presburger};

\draw[->,thick] (3.5, -3) -- (1.25, -4);
\draw[->,thick] (3.5, -3) -- (5.75, -4);

\draw[fill=blue!25] (0, -6) rectangle (7,-7);
\draw (3.5, -6.5) node{PresTaf};

\draw[->,thick] (1.25, -5) -- (3.5, -6);
\draw[->,thick] (5.75, -5) -- (3.5, -6);

\draw[color=orange!75!black] (-2, -0.5) node[align=center]{Bibliothèque};
\draw[color=purple!55!black] (-2, -2.5) node[align=center]{Utilisateur\\de l'outil};
\draw[color=red!75!black] (-2, -4.5) node[align=center]{Utilisateur\\qui fait sa\\propre bibliothèque};
\draw[color=orange!75!black] (-2, -6.5) node[align=center]{Bibliothèque};

\end{tikzpicture}
\caption{Architecture type de PresTaf et son utilisation}
\end{figure}


\section{Description et justification du code}

\subsection{Surcharge d'opérateurs}

L'un des principaux avantages de Lua est d'être un langage de script, basé sur le C++, et donc qui permet la surcharge d'opérateur, 18 au total. Nous avons donc surchargé les opérateurs $+$ $-$ $*$ afin de permettre à l'utilisateur d'utiliser des opérateurs à la place de fonctions pour écrire ses formules Presburger. Ceci donne une écriture plus mathématique, plus fluide. Et surtout cela permet d'avoir une écriture stable et universelle.\\\par

Nous voulions également surcharger l'opérateur $==$, cependant au fil de nos expérimentations, nous nous sommes rendus comptes que nous n'avions aucun contrôle sur le type de retour. Or pour un opérateur comme l'opérateur $==$ ou $~=$ Lua oblige que le retour soit booléen, et ainsi nous n'avons pas pu nous servir de l'opérateur $==$ tel que nous le voulions.\\\par
 
Nous avons donc choisi que cet opérateur ne retournerait pas de valeur mais placerait la valeur dans une variable, il faut donc une deuxième fonction pour l'appeler.\\\par

En Lua, on peut également créer un opérateur personnalisé de la forme $'operateur'$. En effet, lua permet de créer des opérateurs personnalisés en exploitant la fonction $\_\_call$ des métatables. Ces opérateurs doivent être précisés en tant que string. Ces surcharges d’opérateurs permettent à l’utilisateur de la bibliothèque presburger écrite en lua d’avoir plusieurs façons d’écrire une même formule, ce qui lui permet de choisir celle qu’il préfère en fonction de sa convenance. Nous avons choisi de créer l'opérateur $'='$ qui retourne le bon type. L'utilisateur peut donc choisir entre l'opérateur $==$, ce qui lui rajoute une étape de récupération de la variable, et l'opérateur $'='$.

\subsection{L’interface java}

Afin de pouvoir utiliser les fonctionnalités de la bibliothèque PresTaf dans un fichier lua, il faut passer par la bibliothèque luajava, qui permet de récupérer un objet java dans le cadre d'exécution d’un fichier lua. Il a donc fallu créer une classe java qui permet d’interfacer les fonctions de création d’automates afin de pouvoir accéder à ces fonctions dans le lua. Cette classe, appelée 'PresTaf', est donc séparée en deux parties.\\\par 

La première interface est celle qui est directement accessible depuis le lua à travers l’objet lua "prestaf". Cette partie permet d’effectuer des opérations d’égalité (les opérations $=$, $!=$, $<$, $<=$, $>$, $>=$) sur des termes afin d’obtenir une première version des automates associés (ou NPF) de la bibliothèque PresTaf. Cet automate, représentant la formule, est ensuite retourné au lua sous forme d’une deuxième interface. Il est également possible de créer un automate directement en précisant l’état initial, le tableau des successeurs et la liste des états finaux sous forme d’un tableau de booléens.\\\par

La deuxième interface correspond aux fonctions qui permettent de manipuler uniquement un automate associé à une formule. Cette partie est envoyée par la première partie après une opération d’égalité. Cette partie permet d’effectuer des opérations sur les automates telles que les opérations logiques (et, ou, équivalence, implication) ou l’utilisation des quantificateurs (pour tout, il existe). On peut également obtenir une représentation en dot de cet automate.

\subsection{La bibliothèque Presburger Lua}

La bibliothèque Presburger écrite en Lua contient tous les éléments nécessaires pour écrire une formule de Presburger en lua et obtenir l’automate associé. Elle possède les mêmes fonctionnalités que la version précédemment écrite en java. Cette bibliothèque est composée de deux objets lua:

\begin{itemize}
	\item Term: Cet objet permet de construire des variables et des entiers et d’effectuer des opérations élémentaires dessus telles que l’addition, la multiplication et la soustraction. Ces opérateurs ont été surchargés afin d’améliorer la visibilité des formules. Ces opérations construisent un terme en suivant une arborescence. Par exemple, avec le terme $2x + y$ on obtient l’arborescence suivante:
	\begin{figure}[h]
		\centering
		\begin{tikzpicture}
		\node (A) at (0, 0) {$+$};
		\node (B) at (-1, -1) {$*$};
		\node (C) at (1, -1) {$y$};
		\node (D) at (-2, -2) {$2$};
		\node (E) at (0, -2) {$x$};

		\draw (A) -- (B);
		\draw (A) -- (C);
		\draw (B) -- (D);
		\draw (B) -- (E);
		\end{tikzpicture}
	\end{figure}

	\item Presburger: Cet objet utilise deux termes et une opération de comparaison pour se construire.
		Lors de son instanciation, l’objet va convertir les termes en des tableaux reconnus par la bibliothèque PresTaf, puis se servir de cette bibliothèque afin de récupérer l’automate associé. Il est ensuite possible d’effectuer des opérations supplémentaires sur ces formules et d’en obtenir un affichage et une représentation dot.

\end{itemize}

\subsection{Le programme principal}

Le programme principal se lance avec une chaine de caractère représentant le fichier lua à lancer. Ici, la bibliothèque luajava est utilisée afin d’envoyer une instance de l’interface java dans l’environnement lua, puis le fichier lua est exécuté.


\section{Analyser la compléxité}



\subsection{Minimisation d'automate}

Presburger utilise le noyau PresTaf, il est donc directement lié à la complexité de PresTaf. L'une des complexités essentielles de Prestaf est celle de la minimisation d'automate. Un automate peut en effet avoir plusieurs milliers d'états. Nous avons commencé par étudier l'algorithme d'Hopcroft et une révision de Blum qui sont en $O(nlog(n))$. L'algorithme de minimisation d'automate utilisé dans PresTaf s'inspire également d'Hopcroft en utilisant exclusivement des tableaux. La complexité de cet algorithme est donc $O(nlog(n))$. 

\subsection{Quantificateur}

La pluspart des opérations de PresTaf sont exécutées en temps polynomial, cependant lors du calcul d'un quantificateur, la complexité devient une triple exponentielle $O(2^{2^{2^{pn}}})$ selon Derek C. Oppen\cite{oppen1978222pn} où $p$ est une constante $p > 1$ et $n$ est la taille de l'entrée. En effet, l'évaluation d'un quantificateur nécessite de déterminiser l'automate.




\section{Tests de validations et fonctionnement}

\subsection{Test unitaires}

Afin de vérifier la validité de chacune des fonctions écrites, nous avons comparé les resultats des versions Lua et Java.

\subsection{Test globaux}

Nous avons testé la bibliothèque Presburger sur différentes formules. Pour ce faire, nous avons comparé les graphes obtenus par la nouvelle version de Presburger et l'ancienne. Si par exemple on cherche les valeurs de $x$ tel qu'il soit multiple de 3 on aura la forme suivante :

\begin{lstlisting}[mathescape=true, frame=single]
pres = require('Presburger')

local x = variable('x')
local y = variable('y')

local f = x '=' 2 * y 

f:todot("f.dot")
\end{lstlisting}

Le ficher f.dot est le suivant :

\vspace{0.5cm}

\begin{tikzpicture}

\node[circle,draw,accepting] (A) at (0, 4.5)    {$q_0$};
\node[circle,draw,accepting] (B) at (-1.5, 3)   {$q_1$};
\node[circle,draw]           (C) at (-1.5, 1.5) {$q_2$};
\node[circle,draw,accepting] (D) at (0, 0)      {$zero$};

\path[->] (0, 5.5) edge (A)
(A) edge [bend right] node[above]{0} (B)
(B) edge [bend right] node[right]{0} (A)
(B) edge [bend right] node[left]{1}  (C)
(C) edge [bend right] node[right]{1} (B)
(C) edge [bend right] node[above]{0} (D)
(A) edge [bend left]  node[right]{1} (D);
\end{tikzpicture}

Le résultat obtenus avec l'ancienne version de PresTaf est bien entendu la même. En revanche le Lua contient une différence majeure. Si l'on écrit le script suivant :

\begin{lstlisting}[mathescape=true, frame=single]
pres = require('Presburger')

local x = variable('x')
local y = variable('y')

local f = y '=' 2 * x 

f:todot("f.dot")
\end{lstlisting}

Le fichier générer sera différent : \vspace{0.5cm}

\begin{tikzpicture}

\node[circle,draw,accepting] (A) at (0, 4.5)    {$q_0$};
\node[circle,draw,accepting] (B) at (-1.5, 3)   {$q_4$};
\node[circle,draw]           (C) at (1, 3)      {$q_3$};
\node[circle,draw]           (D) at (1.5, 1.5)  {$q_1$};
\node[circle,draw]           (E) at (1, 0)      {$q_5$};
\node[circle,draw]           (F) at (3, 0)      {$q_2$};
\node[circle,draw,accepting] (G) at (0, -1.5)   {$zero$};

\path[->] (0, 5.5) edge (A)
(A) edge [bend right] node[above]{0} (B)
(B) edge [bend right] node[right]{0} (A)
(A) edge              node[left] {1} (C)
(F) edge [bend right] node[right]{1} (A)
(C) edge [bend right] node[left]{1} (G)
(C) edge              node[right]{0} (D)
(D) edge [bend left] node[right]{1} (E)
(E) edge [bend left] node[left]{1} (D)
(B) edge             node[left]{1} (G)
(D) edge             node[above]{0} (F)
(F) edge             node[below]{0} (G)
(E) edge             node[below]{0} (G);
\end{tikzpicture}

Ce qui permet de traiter les variables dans l'ordre désiré, ce qui laisse une plus grande marge de man\oe{}uvre.


\section{Description des extensions possibles}

\subsection{Gestion des erreurs}

Dans la partie 11.1 nous avons expliqué qu'aucune erreur n'était renvoyée. Il serait donc nécessaire de reprendre tout le code de la bibliothèque PresTaf et de Presburger pour afficher directement les erreurs sous forme de chaine de caractères. Cependant, cela implique de prévoir pour chaque fonctions, la plus part des erreurs possibles. De plus, si l'utilisateur appelle une fonction avec les mauvais paramètre, il n'y aurait pas d'affichage


\subsection{Complexité}

Pour améliorer la complexité de Presburger, il est nécessaire d'améliorer celle de la bibliothèque Prestaf. Cependant, les algorithmes utilisés utilisent nombreux tableaux et on été déja pensés pour optimiser le temps d'execution. Il est donc difficile d'améliorer la complexité de PresTaf.

\subsection{Diversité des bibliothèques}

Grâce à PresTaf, il est possible de passer aisément d'une logique a une autre. Nous pouvons par exemple créer simplement l'automate reconnaissant le langage $L = \{x | x = 2^n, n \in \mathbb{N} \}$. En effet, l'automate minimal n'a que 3 états.

\begin{figure}[h]

\centering

\begin{tikzpicture}[->,>=stealth',shorten >=1pt,auto,node distance=1.5cm,
                    semithick]

\node[circle,draw]           (A) at (0, 0) {$q_0$};
\node[circle,draw,accepting] (B) at (2.5, 0)   {$q_1$};
\node[circle,draw]           (C) at (5, 0)  {$q_2$};

\path[->] (A) edge [loop above] node{0} (A)
(A) edge node[above]{1} (B)
(B) edge [loop above] node{0} (B)
(B) edge node[above]{1} (C)
(-1, 0) edge (A)
;

\end{tikzpicture}
\caption{Automate corréspondant à $x = 2^n$}
\end{figure}

Si on associe un ensemble de variables a des booléens, cet automate permet de trouver si une variable est un singleton. Par extension on peut donc passer de la logique Monadique a la logique Presburger grâce à cet automate.


L'objectif du logiciel PresTaf est de permettre de manipuler plusieurs bibliohtèques logiques à partir du même logiciel grâce à la bibliothèque prestaf qui permet de manipuler des automates. Notre mission était de permettre aux utilisateurs de développer simplement de nombreuses extensions.PresTaf ne doit donc pas se limiter à la logique Presburger et devrait être enrichit en développant de nouvelles bibliothèques de différentes logiques telles que la logique monadique. 


\section{Conclusion}

Pour mener a bien ce projet, nous avons utiliser l'outil de versionnage git avec un dépôt gitHub pour favoriser un travail collaboratif et à distance. Nous également créé une java doc pour l'interface de Prestaf. Cette documentation permettra aux utilisateurs de créer plus facilement leur bibliothèque. Nous avons aussi, créé des script bat et sh pour simplifier le lancement du programme et éviter d'entrer une commande javac avec un classpath.
	
Cette applicatation avait plusieurs besoins fonctionnels:
	\begin{itemize}
\item L'utilisateur peut utiliser différentes logiques en appelant simplement la logique qu'il souhaite en début de fichier.
\item Il peut également créer sa propre logique grâce à l'interface de PresTaf et sa javadoc.
\item L'utilisateur peut maintenant écrire ses scripts en Lua pour utiliser Presburger a la place d'un fichier texte traduit par yacc
\item L'application est portable car les deux langages utilisés, java et lua le sont.
\item L'utilisateur a une interface simplifiée grâce aux opérateurs surchargés
\item La pluspart de nos algorithmes n'exedent pas une complexité de $O(n^2)$, l'application reste donc en temps polynomiale ou exponentielle dans le cas des quantificateurs.
\end{itemize}

Ce projet nous a permi de découvrir le langage LUA qui est un langage de script permettant de surcharger les opérateurs. Nous avons également pu découvrir et comprendre des articles de recherches, notamment les algorithmes de Blum et d'Hopcroft qui visent à optimiser la complexité de la minimisation d'automate. Nous avons également appris l'existance de différentes logiques mathématiques telle que la base-2.


\appendix

%\section{Manuel utilisateur}

%Pour implémenter sa propre logique il est nécessaire de suivre une certaine démarche. La démarche sera expliqué avec la logique de Presburger

%\subsection{Fonctions à implémenter}

%Tout d'abord, une logique est un objet Lua qui sera traduit en objet Java pour PresTaf. Pour ce faire vous devrez d'abord créer votre classe Term :

%\lstinputlisting[basicstyle=\tiny]{Term.lua}

%Enfin créez votre logique

%\lstinputlisting[basicstyle=\tiny]{presburger.lua}

\printglossaries

\bibliography{biblio}{}
\addcontentsline{toc}{section}{Références}
\bibliographystyle{plain}

\end{document}
