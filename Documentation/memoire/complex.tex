\section{Analyser la compléxité}

\subsection{Minimisation d'automate}

La bibliohtèque Presburger utilise le noyau PresTaf, elle est donc directement liée à la complexité de PresTaf. L'une des complexités essentielles de Prestaf est celle de la minimisation d'automates. Un automate peut en effet avoir plusieurs milliers d'états. Nous avons commencé par étudier l'algorithme d'Hopcroft et une révision de cet algorithme par Blum qui sont en $O(nlog(n))$. L'algorithme de minimisation d'automates utilisé dans PresTaf s'inspire également d'Hopcroft mais utilise exclusivement des tableaux comme structure de données. La complexité de cet algorithme est donc $O(nlog(n))$. 

\subsection{Quantificateur}

La plupart des opérations de PresTaf sont exécutées en temps polynomial, cependant lors du calcul d'un quantificateur, la complexité devient une triple exponentielle $O(2^{2^{2^{pn}}})$ selon Derek C. Oppen\cite{oppen1978222pn} où $p$ est une constante $p > 1$ et $n$ est la taille de l'entrée. En effet, l'évaluation d'un quantificateur nécessite de déterminiser l'automate.
