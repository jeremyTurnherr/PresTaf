\section{Description des extensions possibles}

\subsection{Complexité}

Pour améliorer la complexité de Presburger, il est nécessaire d'améliorer celle de la bibliothèque Prestaf. Cependant, les algorithmes utilisés travaillent sur de nombreux tableaux et on été déja été pensés par monsieur Couvreur pour optimiser le temps d'execution. Il est donc difficile d'améliorer la complexité de PresTaf.

\subsection{Diversité des bibliothèques}

Grâce à PresTaf, il est possible de passer aisément d'une logique à une autre. Nous pouvons par exemple créer simplement l'automate reconnaissant le langage $L = \{x | x = 2^n, n \in \mathbb{N} \}$. En effet, l'automate minimal n'a que 3 états.

\begin{figure}[h]

\centering

\begin{tikzpicture}[->,>=stealth',shorten >=1pt,auto,node distance=1.5cm,
                    semithick]

\node[circle,draw]           (A) at (0, 0) {$q_0$};
\node[circle,draw,accepting] (B) at (2.5, 0)   {$q_1$};
\node[circle,draw]           (C) at (5, 0)  {$q_2$};

\path[->] (A) edge [loop above] node{0} (A)
(A) edge node[above]{1} (B)
(B) edge [loop above] node{0} (B)
(B) edge node[above]{1} (C)
(-1, 0) edge (A)
;

\end{tikzpicture}
\caption{Automate corréspondant à $x = 2^n$}
\end{figure}

Si on associe un ensemble de variables à des booléens, cet automate permet de trouver si une variable est un singleton. Par extension on peut donc passer de la logique Monadique à la logique Presburger grâce à cet automate.\\\par

L'objectif du logiciel PresTaf est de permettre de manipuler plusieurs bibliothèques logiques à partir du même logiciel grâce à la bibliothèque PresTaf qui permet de manipuler des automates. Notre mission était de permettre aux utilisateurs de développer simplement de nombreuses extensions.PresTaf ne doit donc pas se limiter à la logique Presburger et devrait être enrichi en développant de nouvelles bibliothèques de différentes logiques telles que la logique monadique. 
