\section{Analyse de l'existant}

PresTaf est un programme codé par M. Jean-Michel Couvreur, qui prend des formules de Presburger en entrées et les résout à l'aide d'automates minimaux. Tout d'abord il génère des automates déterministes et finis mais non minimaux. Il faut donc les minimiser, et pour se faire PresTaf utilise un algorithme d'Hopcroft modifié. L'ensemble des transitions menant de l'état initial vers un état final est solution de la formule. En outre si l'état final est l'état \emph{zero} alors il n'y aucune solution et si l'état final est l'état \emph{one} alors la formule est une tautologie.\\\par

Mona, est une bibliothèque C qui résout des formules monadique. La où PresTaf n'implémente à ce jour que la logique arithmétique de Presburger, la logique monadique pourrait être implémenter dans le futur.\\\par

Lash\cite{lash} est une bibliothèque C qui résout des formules de Presburger, mais la différence avec PresTaf est qu'il fonctionne sur des automates infini. Cette différence induit une importante baisse de performante. En effet PresTaf pour les mêmes formules était bien plus rapide à s'executer que Lash\cite{DBLP:conf/wia/Couvreur04}.\\\par
