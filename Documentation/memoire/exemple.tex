\section{Fonctionnement}

\subsection{Utilisation de Presburger}

Si l'on souhaite connaitre l'automate qui reconnait $L = \{x | x = 2*y, y \in \mathbb{N} \}$ on peut l'écrire de la façon suivante avec PresTaf.\\


\begin{lstlisting}[mathescape=true, frame=single]
pres = require('Presburger')

local x = variable('x')
local y = variable('y')

local f = x '=' 2 * y 

f:todot("f.dot")
\end{lstlisting}

Le ficher f.dot obtenu est le suivant :

\vspace{0.5cm}

\begin{tikzpicture}

\node[circle,draw,accepting] (A) at (0, 4.5)    {$q_0$};
\node[circle,draw,accepting] (B) at (-1.5, 3)   {$q_1$};
\node[circle,draw]           (C) at (-1.5, 1.5) {$q_2$};
\node[circle,draw,accepting] (D) at (0, 0)      {$zero$};

\path[->] (0, 5.5) edge (A)
(A) edge [bend right] node[above]{0} (B)
(B) edge [bend right] node[right]{0} (A)
(B) edge [bend right] node[left]{1}  (C)
(C) edge [bend right] node[right]{1} (B)
(C) edge [bend right] node[above]{0} (D)
(A) edge [bend left]  node[right]{1} (D);
\end{tikzpicture}\\

On obtient le même résultat si on entre cette formule dans l'ancienne application Presburger. En revanche le Lua contient une différence majeure. En effet, si l'utilisateur écrit le script suivant :

\begin{lstlisting}[mathescape=true, frame=single]
pres = require('Presburger')

local x = variable('x')
local y = variable('y')

local f = y '=' 2 * x 

f:todot("f.dot")
\end{lstlisting}

Le fichier généré sera différent : \vspace{0.5cm}

\begin{tikzpicture}

\node[circle,draw,accepting] (A) at (0, 4.5)    {$q_0$};
\node[circle,draw,accepting] (B) at (-1.5, 3)   {$q_4$};
\node[circle,draw]           (C) at (1, 3)      {$q_3$};
\node[circle,draw]           (D) at (1.5, 1.5)  {$q_1$};
\node[circle,draw]           (E) at (1, 0)      {$q_5$};
\node[circle,draw]           (F) at (3, 0)      {$q_2$};
\node[circle,draw,accepting] (G) at (0, -1.5)   {$zero$};

\path[->] (0, 5.5) edge (A)
(A) edge [bend right] node[above]{0} (B)
(B) edge [bend right] node[right]{0} (A)
(A) edge              node[left] {1} (C)
(F) edge [bend right] node[right]{1} (A)
(C) edge [bend right] node[left]{1} (G)
(C) edge              node[right]{0} (D)
(D) edge [bend left] node[right]{1} (E)
(E) edge [bend left] node[left]{1} (D)
(B) edge             node[left]{1} (G)
(D) edge             node[above]{0} (F)
(F) edge             node[below]{0} (G)
(E) edge             node[below]{0} (G);
\end{tikzpicture}

Ce qui permet de traiter les variables dans l'ordre désiré, et laisse ainsi, une plus grande marge de man\oe{}uvre.\\\par

Pour utiliser la bibliothèque Presburger l'utilisateur doit l'importer via la fonction $require()$ de lua pour pouvoir appeler les fichiers Term et Presburger.\\\par

Il existe les fonctions $variable(chaine)$ et $integer(nombre)$ qui permettent de créer respectivement une variable et un nombre. Ces deux élements peuvent ensuite être manipulés à l'aide de plusieurs fonctions qui implémentent la logique de Presburger.\\\par

La plupart de ces fonctions peuvent être également appelées directement par des opérateurs surchargés, de manière à ce que l'utilisateur puisse avoir une écriture plus lisible et concise pour de très longue formules Presburger. On trouve ainsi avec les opérateurs:
\begin{itemize}
 \item les opérations $factor()$, $plus()$ et $minus()$, \{*,+,-\}
 \item les quantificateurs $\_A()$ et $\_E()$,{'A','E'}
 \item les comparateurs $equals()$, $not\_Equals()$, $greater()$,\\$greaterEquals()$, $lessEquals()$, $less()$,\{ $'='$, $!=$, $>$, $>=$, $<=$, $<=$, $<$\}
 \item les opérateurs logiques $and()$, $or()$, $not()$, $equiv()$ et $imply()$.\{ $\&\&$, $||$, $<->$, $->$\}
\end{itemize}
Voici un exemple d'utilisation pour la formule $6x + 4 = 1$\\
 
\begin{lstlisting}[mathescape=true, frame=single]
pres = require('Presburger')
local x = variable('x')
local f=(6*x+integer(4))'='(integer(1))
f:todot("f.dot")
\end{lstlisting}

Pour exprimer qu'une formule est toujours vraie ou fausse, PresTaf renvoit respectivement les états one ou zero.
\begin{figure}[h]
\begin{lstlisting}[mathescape=true, frame=single]
pres = require('Presburger')
local x = variable('x')
local f=(_E(x,(x)'='(integer(1)))
f:todot("f.dot")
\end{lstlisting}

\begin{tikzpicture}[->,>=stealth',shorten >=1pt,auto,node distance=1.5cm,
                    semithick]

\node[circle,draw,accepting]           (A) at (0, 0) {$one$};

\end{tikzpicture}
\caption{Formule toujours vraie}
\end{figure}


\begin{lstlisting}[mathescape=true, frame=single]
pres = require('Presburger')
local x = variable('x')
local f=(_A(x,(x)'='(integer(1)))
f:todot("f.dot")
\end{lstlisting}

\begin{figure}[h]
\begin{tikzpicture}[->,>=stealth',shorten >=1pt,auto,node distance=1.5cm,
                    semithick]

\node[circle,draw,accepting]           (A) at (2, 2) {$zero$};

\end{tikzpicture}
\caption{Formule toujours fausse}
\end{figure}

\subsection{ Developpement d’une bibliothèque logique}

Pour créer une nouvelle bibliothèque qui implémente une logique en utilisant l’interface PresTaf, il est nécessaire de créer des fonctions ou des métatables lua afin de contenir et de pouvoir modifier les différentes informations, notamment les coefficients des variables et objectifs. Dans l’exemple des termes de Presburger, la métatable lua Term contient ces informations en tant qu’attribut (var pour les numéros de variable, coef pour les coefficients associés et constant pour la constante objectif).\par

Il est important de noter qu’il n’est pas nécessaire d’avoir l’indice des variables pour créer un automate. En effet, les indices des variables peuvent correspondre aux indices de leurs coefficients.\par

Ensuite il suffit de créer les opérations élémentaires que l’on souhaite effectuer sur la logique que l’on veut implémenter. Ces opérations doivent impacter la liste des coefficients des variables ou la constante.\par

Enfin, pour générer l’automate, il faut appeler les opérations de comparaisons de la bibliothèque prestaf afin d’obtenir l’automate associé à la formule que l’on a écrite. Il est possible d’effectuer des opérations supplémentaires sur cet automate comme d'appliquer des opérateurs logiques sur deux formules.\\\par

Il est également possible de créer entièrement un automate en précisant l’état initial, une liste de booléens représentant les états finaux et le tableau des successeurs en fonction des lettres de l’alphabet. Pour cela, il faut passer par le fichier ‘Automaton.lua’ ou la fonction associée est présente.
