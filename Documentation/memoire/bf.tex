\section{Besoins fonctionnels}

\subsection{Prototype papier}

Le prototype qui suit serait un fichier Lua qui se servirait de la logique $Presburger$ (cette logique étant elle-même codée en Lua).

\begin{lstlisting}[mathescape=true, frame=single]
pres = require('Presburger') // Choix de la logique

local x = variable('x') // Declaration d'une variable
local y = variable('y')
local f = equals(y + integer(1), x)

// Pour exporter l'automate
f:todot("f.dot")
\end{lstlisting}

Pour lancer ce fichier Lua il faudrait passer par un fichier jar que l'on appellera $prestaf.jar$. Pour lancer le jar et le fichier lua il faudrait faire la commande suivante :

\begin{lstlisting}[mathescape=true, frame=single]
java -jar prestaf.jar fichier.lua
\end{lstlisting}

Le fichier f.dot ressemblerai à :

\begin{tikzpicture}
\node[circle,draw]           (A) at (0.75, 6)  {$q_0$};
\node[circle,draw]           (B) at (0, 4.5)   {$q_1$};
\node[circle,draw,accepting] (C) at (1.5, 4.5) {$q_2$};
\node[circle,draw,accepting] (D) at (2, 3)     {$q_3$};
\node[circle,draw]           (E) at (2, 1.5)   {$q_4$};
\node[circle,draw,accepting] (F) at (0.75, 0)  {$zero$};

\path[->] (0.75, 7) edge (A)
(A) edge [bend right] node[above]{0} (B)
(B) edge [bend right] node[right]{1} (A)
(A) edge [bend left] node[right]{1} (C)
(B) edge [bend right] node[right]{0} (F)
(C) edge [bend right] node[right]{1} (F)
(C) edge [bend right] node[right]{0} (D)
(D) edge [bend right] node[right]{0} (C)
(D) edge [bend right] node[right]{1} (E)
(E) edge [bend right] node[right]{1} (D)
(E) edge [bend left] node[right]{0} (F);
\end{tikzpicture}

\subsection{Fonctions}

La priorité des fonctions varie de 1 à 5, du plus important au moins important, sachant que la priorité 5 correspond à une fonctionnalité optionnelle.

\subsubsection{Bilbiothèque d'automate générique}

\paragraph{Description :} La bibliothèque PresTaf est générique et doit accepter toutes sortes d'automates.

\paragraph{Justification :} L'utilisateur aura la possibilité d'implémenter ses propres logiques, la bibliothèque doit donc accepter toutes logiques. En effet PresTaf sera une bibliothèque d'automates, permettant de minimiser un automate, de faire des intersections, des unions, etc. Il ne faut donc pas que PresTaf soit ciblé sur Presburger, mona ou une quelconque autre logique.

\paragraph{Priorité :} 1\\

\rule{\linewidth}{1pt}

\subsubsection{Minimisation d'automate}

\paragraph{Description :} Ensemble de fonctions qui prennent un automate (fini, complet) et déterministe en entrée et retourne l'automate minimal équivalent.

\paragraph{Justification :} Besoin initial.

\paragraph{Priorité :} 1\\

\rule{\linewidth}{1pt}

\subsubsection{Interfaçage Lua}

\paragraph{Description :} Permet le codage des automates en Lua, ainsi que l'utilisation de chaques fonctions qui seront ensuite exécutées en Java.

\paragraph{Justification :} Le lua est un langage de script simple à prendre en main et qui permet facilement d'écrire des automates et d'utiliser des fonctions.

\paragraph{Priorité :} 2\\

\rule{\linewidth}{1pt}

\subsubsection{Portabilité du code}

\paragraph{Description :} Windows, MacOS, Linux

\paragraph{Justification :} Comme java est un langage portable executé via la Java Virutal Machine (JVM), et que LuaJava est executé via java il embarque sa propre machine virtuelle, en théorie le code sera donc portable sur tous les systèmes d'exploitation. En dehors de la portabilité du code, Windows MacOs et Linux sont les principaux systèmes d'exploitation, il est donc important d'avoir un code portable pour chaque machine pour faciliter l'accès.

\paragraph{Priorité :} 1\\

\rule{\linewidth}{1pt}

\subsubsection{Optimisation}

\paragraph{Description :} La bilbiothèque d'automates PresTaf doit être rapide à s'exécuter.

\paragraph{Justification :} L'utilisateur n'aura pas le temps d'attendre quelques dizaines de minutes que son automate soit généré. Il voudra obtenir son résultat rapidement.

\paragraph{Priorité :} 5
