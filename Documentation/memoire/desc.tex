\section{Description et justification du code}

\subsection{Surcharge d'opérateurs}

L'un des principaux avantages de Lua est d'être un langage de script, basé sur le C++ et donc qui permet la surcharge d'opérateur, 18 au total. Nous avons donc surchargé les opérateurs $+$ $-$ $*$ afin de permettre à l'utilisateur d'utiliser des opérateurs à la place de fonctions pour écrire ses formules Presburger. Ceci donne une écriture plus mathématiques, plus fluide. Et surtout ça permet d'avoir une écriture stable et universelle.\\\par

Nous voulions également surcharger l'opérateur $==$, cependant au fil de nos expérimentations, comme Lua est un langage typé dynamiquement, nous n'avons aucun contrôle sur le type de retour. Or pour un opérateur comme l'opérateur $==$ ou $~=$ Lua oblige que le retour soit booléen, et ainsi nous n'avons pas pu se servir de l'opérateur $==$ tel que nous le voulions.\\\par
 
Nous avons donc choisi que cet opérateur ne retournerait pas de valeur mais placerait la valeur dans une variable, il faut donc une deuxieme fonction pour l'appeler.\\\par

En Lua, on peut également créé un opérateur pérsonalisé de la forme $'operateur'$. Nous avons choisi de créer l'opérateur $'='$ qui retourne le bon type. L'utilisateur peut donc choisir entre l'opérateur $==$, ce qui lui rajoute une étape de récupération de la variable, et l'opérateur $'='$.
