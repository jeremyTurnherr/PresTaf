\section{Description et justification du code}

\subsection{Surcharge d'opérateurs}

L'un des principaux avantages de Lua est d'être un langage de script qui permet la surcharge d'opérateur. On peut en sucharger 18. Nous avons donc surchargé les opérateurs $+$ $-$ $*$ afin de permettre à l'utilisateur d'utiliser des opérateurs à la place de fonctions pour écrire ses formules Presburger.\par
Nous voulions également surcharger l'opérateur $==$, cependant nous nous sommes rendus comptes que quelque soit la valeur de retour que nous mettions, elle est était toujours transformée en booléen.
Nous avons donc choisi que cet opérateur ne retournerait pas de valeur mais placerait la valeur dans une variable, il faut donc une deuxieme fonction pour l'appeler.\par
En Lua, on peut également créé un opérateur de la forme $'operateur'$. Nous avons choisi de créer l'opérateur $'='$ qui retourne la bonne valeur. L'utilisateur peut donc choisir entre l'opérateur $==$ et l'opérateur $'='$.\par
Pour l'égalité, l'addition, la soustraction et la multiplication de Presburger, l'utilisateur peut donc utiliser la fonction associée ou utiliser l'opérateur surchargé.
